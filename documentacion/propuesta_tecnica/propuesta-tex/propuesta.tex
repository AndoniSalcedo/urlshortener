\documentclass{article}
\usepackage[utf8]{inputenc}
\usepackage[spanish]{babel}
\usepackage{graphicx}
\usepackage{float}
\usepackage{adjustbox}
\usepackage{subcaption}    
\usepackage{geometry}
\usepackage{tabulary}

\geometry{
   a4paper,
   total={170mm,257mm},
   left=35mm,
   right=35mm,
   top=30mm,
   bottom=30mm,
}
 
\title{Propuesta de acuerdo del proyecto}
\author{Ingeniería web\\\\\textbf{Lider}: Salcedo Navarro, Andoni (785649)\\Pelayo Benedet, Tomás (779691)\\Subías Rodríguez, Rubén (759406)\\Velasco Calvo, Isaac (758986)\\Grupo 13}
\date{Octubre 2021}
\setlength{\parindent}{2em}
\setlength{\parskip}{1em}
 
\begin{document}
 
\maketitle
 
\pagebreak
 
%// TODO: pequeña introduccion para cada apartado
\section*{Objetivos}

% // TODO: hacer objetivos

\begin{itemize}
    \item O1.
    \item O2.
    \item 
\end{itemize}

\section*{Funcionalidades}


% // TODO: relacionar los objetivos con las funcionalidades y poner en los parentesis que objetivos se conectan con cada funcionalidad

\begin{itemize}
    \item F1. La aplicación implementará un chat grupal en el que se podrán compartir enlaces con otras personas que usen la aplicación (). Con el objetivo de aprender como implementar canales de comunicación bidireccionales.
    \item F2. La aplicación soportará un inicio de sesión(). Con el objetivo de que cada usuario pueda acceder a sus uri's creadas para poder modificarlas, compartirlas, añadir, eliminar, monetizar, con la finalidad de entender como se realiza el control de sesión y gestión de contraseñas en un nuevo framework.
    \item F3. Se incluye un sistema de redirección para que el usuario pueda monetizar el número de clicks sobre sus uri's (). Con el objetivo de entender como se crea un modelo de negocio real a partir de una aplicación.
    \item F4. La aplicación verificará que la uri que el usuario quiere crear es alcanzable (). El objetivo es aprender como se puede comprobar si una redirección es valida, para evitar redirecciones dañinas o inexistentes.
\end{itemize}

% //TODO: debatir los pesos que debe tener cada funcionalidad y poner los objetivos de cada una

\begin{table}[hbtp]
    \footnotesize
    \centering
    \settowidth\tymin{\textbf{Activities}}
    \setlength\extrarowheight{5pt}
    \begin{tabulary}{\textwidth}{ C C L C}
        \textbf{ID} & 
        \textbf{Peso} & 
        \textbf{Funcionalidad} & 
        \textbf{Objetivos}\\
    \hline
    \hline
        F1 & 
        X & 
        La aplicación implementará un chat grupal en el que se podrán compartir enlaces con otras personas que usen la aplicación. &
        Ox, Oy\\
    \hline  
        F2 &
        X &
        La aplicación soportará un inicio de sesión. &
        Ox, Oy\\
    \hline
        F3 &
        X &
        Se incluye un sistema de redirección para que el usuario pueda monetizar el número de clicks sobre sus uri's. &
        Ox, Oy\\
    \hline
        F4 &
        X &
        La aplicación verificará que la uri que el usuario quiere crear es alcanzable. &
        Ox, Oy

    \end{tabulary}
    \caption{Tabla de reparto de pesos.}
\end{table}

\section*{Evalución de las funcionalidades}
% //TODO: definir un metodo de evaluación de las funcionalidades
\section*{Orientaciones sobre la descripción de funcionalidades}

% //TODO: definir que objetivos de funcionalidad escalabilidad y profesionalidad necesita cada funcionalidad

\pagebreak

\begin{table}[hbtp]
    \footnotesize
    \centering
    \settowidth\tymin{\textbf{Funcionalidad}}
    \setlength\extrarowheight{5pt}
    \begin{tabulary}{\textwidth}{ C L }
        \textbf{F1} & La aplicación implementará un chat grupal en el que se podrán compartir enlaces con otras personas que usen la aplicación.
        \\
    \hline
    
    Funcionalidad & adfadfasdfasdf\\
        
    Escalabilidad & fafadfa \\

    Profesionalidad & dfasdfadsasdf \\

    \end{tabulary}
\end{table}

\begin{table}[hbtp]
    \footnotesize
    \centering
    \settowidth\tymin{\textbf{Funcionalidad}}
    \setlength\extrarowheight{5pt}
    \begin{tabulary}{\textwidth}{ C L }
        \textbf{F2} & La aplicación soportará un inicio de sesión.
        \\
    \hline
    
    Funcionalidad & adfadfasdfasdf\\
        
    Escalabilidad & fafadfa \\

    Profesionalidad & dfasdfadsasdf \\

    \end{tabulary}
\end{table}

\begin{table}[hbtp]
    \footnotesize
    \centering
    \settowidth\tymin{\textbf{Funcionalidad}}
    \setlength\extrarowheight{5pt}
    \begin{tabulary}{\textwidth}{ C L }
        \textbf{F3} & Se incluye un sistema de redirección para que el usuario pueda monetizar el número de clicks sobre sus uri's.
        \\
    \hline
    
    Funcionalidad & adfadfasdfasdf\\
        
    Escalabilidad & fafadfa \\

    Profesionalidad & dfasdfadsasdf \\

    \end{tabulary}
\end{table}

\begin{table}[hbtp]
    \footnotesize
    \centering
    \settowidth\tymin{\textbf{Funcionalidad}}
    \setlength\extrarowheight{5pt}
    \begin{tabulary}{\textwidth}{ C L }
        \textbf{F4} & La aplicación verificará que la uri que el usuario quiere crear es alcanzable. 
        \\
    \hline
    
    Funcionalidad & Solo se puede crear una URI acortada si se verifica que una petición HTTP GET
    a la URI original devuelve una respuesta con estado 200.\\
        
    Escalabilidad & crea la URI acortada pero no se activa; en un hilo de ejecución separado se
    verifica que la URI original es alcanzable mediante una petición http GET que debe devolver
    una respuesta con estado 200 en un tiempo máximo configurable; si es alcanzable la URI
    acortada se activa, si no lo es la URI acortada se marca como inválida. \\

    Profesionalidad & toda la funcionalidad está cubierta por pruebas automáticas de unidad u de
    integración que comprueban al menos dos casos (URI alcanzable, URI no alcanzable), el API
    relacionado está documentado utilizando Open API, hay una consola Web interactiva que
    visualiza dicha documentación y que permite probar dicha API y hay documentación en
    AsciiDoc explicando el funcionamiento del código que ha sido convertida a HTML5 con
    Asciidoctor. \\

    \end{tabulary}
    
\end{table}
 
\end{document}
