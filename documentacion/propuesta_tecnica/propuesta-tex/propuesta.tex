\documentclass{article}
\usepackage[utf8]{inputenc}
\usepackage[spanish]{babel}
\usepackage{graphicx}
\usepackage{float}
\usepackage{adjustbox}
\usepackage{subcaption}    
\usepackage{geometry}
\usepackage{tabulary}

\geometry{
   a4paper,
   total={170mm,257mm},
   left=35mm,
   right=35mm,
   top=30mm,
   bottom=30mm,
}
 
\title{Propuesta de acuerdo del proyecto}
\author{Ingeniería web\\\\\textbf{Lider}: Salcedo Navarro, Andoni (785649)\\Pelayo Benedet, Tomás (779691)\\Subías Rodríguez, Rubén (759406)\\Velasco Calvo, Isaac (758986)\\Grupo 13}
\date{Octubre 2021}
\setlength{\parindent}{2em}
\setlength{\parskip}{1em}
 
\begin{document}
 
\maketitle
 
\pagebreak
 
\section*{Objetivos}

\begin{itemize}
    \item O1.
    \item O2.
    \item 
\end{itemize}

\section*{Funcionalidades}

\begin{itemize}
    \item F1. La aplicación implementará un chat grupal en el que se podrán compartir enlaces con otras personas que usen la aplicación. 
    \item F2. La aplicación soportará un inicio de sesión con el objetivo de que cada usuario pueda acceder a sus uri's creadas para poder modificarlas, compartirlas, añadir, eliminar, monetizar.
    \item F3. Se incluye un sistema de redirección para que el usuario pueda monetizar el número de clicks sobre sus uri's.
    \item F4. La aplicación verificará que la uri que el usuario quiere crear es alcanzable.
\end{itemize}


\begin{table}[hbtp]
    \footnotesize
    \centering
    \settowidth\tymin{\textbf{Activities}}
    \setlength\extrarowheight{5pt}
    \begin{tabulary}{\textwidth}{ C C C C}
        \textbf{ID} & 
        \textbf{Peso} & 
        \textbf{Funcionalidad} & 
        \textbf{Objetivos}\\
    \hline
    \hline
        F1 & 
        X & 
        La aplicación implementará un chat grupal en el que se podrán compartir enlaces con otras personas que usen la aplicación. &
        Ox, Oy\\
    \hline  
        F2 &
        X &
        La aplicación soportará un inicio de sesión con el objetivo de que cada usuario pueda acceder a sus uri's creadas para poder modificarlas, compartirlas, añadir, eliminar, monetizar. &
        Ox, Oy\\
    \hline
        F3 &
        X &
        Se incluye un sistema de redirección para que el usuario pueda monetizar el número de clicks sobre sus uri's. &
        Ox, Oy\\
    \hline
        F4 &
        X &
        La aplicación verificará que la uri que el usuario quiere crear es alcanzable. &
        Ox, Oy

    \end{tabulary}
    \caption{Tabla de reparto de pesos.}
    \label{XXX}
\end{table}

 
\end{document}
