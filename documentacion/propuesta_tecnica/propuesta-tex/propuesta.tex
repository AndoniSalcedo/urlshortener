\documentclass{article}
\usepackage[utf8]{inputenc}
\usepackage[spanish]{babel}
\usepackage{graphicx}
\usepackage{float}
\usepackage{adjustbox}
\usepackage{subcaption}    
\usepackage{geometry}
\usepackage{tabulary}
\usepackage{tabto}
 
\geometry{
   a4paper,
   total={170mm,257mm},
   left=35mm,
   right=35mm,
   top=30mm,
   bottom=30mm,
}
 
\title{Propuesta de acuerdo del proyecto}
\author{Ingeniería web\\\\Grupo 13}
\date{Octubre 2021}
\setlength{\parindent}{2em}
\setlength{\parskip}{1em}
 
\begin{document}
 
\maketitle
 
\begin{figure}[H]
    \captionsetup[subfigure]{justification=centering}
     \begin{subfigure}[b]{0.49\textwidth}
        \centering
         \includegraphics[width=0.5\textwidth]{../images/Andoni.jpeg}
         \caption*{\textbf{Líder}: Salcedo Navarro, Andoni (785649)}
     \end{subfigure}
     \hfill
     \begin{subfigure}[b]{0.49\textwidth}
        \centering
        \includegraphics[width=0.5\textwidth]{../images/Tomas.jpeg}
        \caption*{Pelayo Benedet, Tomás (779691)}
     \end{subfigure}
     \hfill
     \begin{subfigure}[b]{0.49\textwidth}
        \centering
        \includegraphics[width=0.5\textwidth]{../images/Ruben.jpeg}
        \caption*{Subías Rodríguez, Rubén (759406)}
     \end{subfigure}
     \hfill
     \begin{subfigure}[b]{0.49\textwidth}
        \centering
        \includegraphics[width=0.5\textwidth]{../images/Isaac.jpeg}
        \caption*{Velasco Calvo, Isaac (758986)}
     \end{subfigure}
 
    \end{figure}
 
\pagebreak
 
\section*{Objetivos}
 
Se han analizado las características del equipo de trabajo y los requerimientos que son necesarios para afrontar el proyecto y se ha llegado a la conclusión de que los objetivos del equipo de cara al proyecto son los siguientes:
 
\begin{itemize}
    \item O1. Ser capaz de identificar las necesidades para desarrollar una aplicación Web que se pueda desplegar en producción.
    \item O2. Ser capaz de adaptar las nuevas funcionalidades propuestas y desplegarlas en el sistemas web ya funcional.
    \item O3. Comprender cómo funcionan las pasarelas publicitarias para poder monetizar la aplicación.
    \item O4. Comprensión de las arquitectura detrás de los  canales de comunicación multicast.
    \item O5. Entender cómo gestionar el almacenamiento y encriptado de datos sensibles.
    \item O6. Aprender a utilizar los conceptos de UX/UI para que la usabilidad de la aplicación sea la adecuada para el usuario final.
\end{itemize}
 
\section*{Funcionalidades}
 
\begin{itemize}
    \item F1. La aplicación dara la capacidad de compartir enlaces con otras personas a través de redes sociales (O1,O2,O4). Con el objetivo de aprender cómo implementar canales de comunicación bidireccionales.
    \item F2. La aplicación soportará un inicio de sesión y permitirá al usuario gestionar sus URL's asociadas así como el número de usos. (O1,O2,O5,O6). Con el objetivo de que cada usuario pueda acceder a sus uri 's creadas para poder modificarlas, compartirlas, añadir, eliminar, monetizar, con la finalidad de entender cómo se realiza el control de sesión y gestión de contraseñas en un nuevo framework.
    \item F3. Se incluye un sistema de redirección para que el usuario pueda monetizar. (O1,O3). Con el objetivo de entender cómo se crea un modelo de negocio real a partir de una aplicación.
    \item F4. La aplicación verificará que la uri que el usuario quiere crear es alcanzable (O2,O5). El objetivo es aprender cómo se puede comprobar si una redirección es válida, para evitar redirecciones dañinas o inexistentes.
\end{itemize}
 
\begin{table}[hbtp]
    \footnotesize
    \centering
    \settowidth\tymin{\textbf{Activities}}
    \setlength\extrarowheight{5pt}
    \begin{tabulary}{\textwidth}{ C C L C}
        \textbf{ID} & 
        \textbf{Peso} & 
        \textbf{Funcionalidad} & 
        \textbf{Objetivos}\\
    \hline
    \hline
        F1 & 
        15 & 
        La aplicación dara la capacidad de compartir enlaces con otras personas a través de redes sociales. &
        O1, O2, O4\\
    \hline  
        F2 &
        15 &
        La aplicación soportará un inicio de sesión y permitirá al usuario gestionar sus URL's asociadas así como el número de usos. &
        O1, O2, O5, O6\\
    \hline
        F3 &
        15 &
        Se incluye un sistema de redirección para que el usuario pueda monetizar. &
        O1, O3\\
    \hline
        F4 &
        15 &
        La aplicación verificará que la uri que el usuario quiere crear es alcanzable. &
        O2, O5
 
    \end{tabulary}
    \caption{Tabla de reparto de pesos.}
\end{table}
 
\pagebreak
 
\section*{Orientaciones sobre la descripción de funcionalidades}
 
 
\begin{table}[hbtp]
    \footnotesize
    \centering
    \settowidth\tymin{\textbf{Funcionalidad}}
    \setlength\extrarowheight{5pt}
    \begin{tabulary}{\textwidth}{ C L }
        \textbf{F1} & La aplicación dara la capacidad de compartir enlaces con otras personas a través de redes sociales.
        \\
    \hline
    
    Cumplimiento &
     \labelitemi \quad Cuando un usuario cree una URI acortada se le proporcionará además una URI que devolverá un código QR que contendrá codificada la URI acortada &&

     \labelitemi \quad Si esta funcionalidad se implementa con un API Web su implementación será consistente con la semántica de http. &&
     
     implica: &&
     \labelitemi \quad POST /short y POST /user/short soportará un parámetro opcional por el que se indicará si habrá o no una representación en código QR de la URI acortada. Si
     no está presente se entenderá que no habrá código QR. La respuesta
     http contendrá en el JSON una propiedad que tendrá la dirección URI
     completa del código QR  &&
     
     \labelitemi \quad GET /qr devolverá el QR correspondiente a una dirección URL en caso de estar marcado el campo correspondiente a la representación del codigo QR\\
        
    Escalabilidad & 
    5: && 
        \quad \labelitemi \quad No necesita escalar &&
    10: &&
        \quad \labelitemi \quad Cumplir nivel 5 && 
        \quad \labelitemi \quad La URL del usuario se compartirá con el resto de usuarios de la aplicación a través de correo electrónico. && 
    15: &&
        \quad \labelitemi \quad Cumplir nivel 10 &&
        \quad \labelitemi \quad La url de usuario se autopublicará en Twitter.\\
 
    Profesionalidad &
    
    Si esta funcionalidad se implementa con un API Web, la parte expuesta al usuario está documentado con Open API 3.0 AND El API expuesto al usuario tendrá pruebas automáticas que verificarán el cumplimiento AND El código del API expuesto al usuario estará documentado\\
 
    \end{tabulary}
\end{table}
 
\begin{table}[hbtp]
    \footnotesize
    \centering
    \settowidth\tymin{\textbf{Funcionalidad}}
    \setlength\extrarowheight{5pt}
    \begin{tabulary}{\textwidth}{ C L }
        \textbf{F2} & La aplicación soportará un inicio de sesión y permitirá al usuario gestionar sus URL's asociadas así como el número de usos.
        \\
    \hline
    
    Cumplimiento &
        \labelitemi \quad El usuario puede registrarse en la aplicación. &&
        \labelitemi \quad Estas nuevas URIs generadas deben de guardarse en el servidor Web, y también se tiene que guardar que usuario la ha generado. &&
        \labelitemi \quad Opcionalmente, el usuario puede decidir que debe recibir "puntos" en su cuenta por cada vez que una de las URIs generadas por este es usada.
        
        implica:

        \labelitemi \quad POST /register soportará tres parametros nombre, email y contraseña. Se guardará el usuario especificado en la base de datos &&

        \labelitemi \quad POST /login soportará dos parametros, email y contraseña, si las credenciales coenciden con un usuario en almacenado en la base de datos, se iniciara una sesión. &&

        \labelitemi \quad POST /user/shorter soportará los mismos parametros que soporta /shorter y además asignará esa url al usuario, Requiere establecer sesión. 
        \\
        
    Escalabilidad &
    5: &&
        \quad \labelitemi \quad No necesita escalar. &&
         
    10: &&
        \quad \labelitemi \quad Cumplir nivel 5. && 

        \quad \labelitemi \quad La información que se necesita se obtiene previamente, sesión de usuario persistente. &&
		
        \quad \labelitemi \quad La gestión de la sesión es relegada al usuario para ahorrar tiempo de procesamiento al servidor &&
    15: &&
        \quad \labelitemi \quad Cumplir nivel 10. && 
        \quad \labelitemi \quad La base de datos es desplegada a través de un servicio externo que proporcione escalabilidad, persistencia y tolerancia a fallos.
         \\
 
    Profesionalidad & 
    \labelitemi \quad El API expuesto al usuario tendrá pruebas automáticas que verificarán el cumplimiento. &&
    \labelitemi \quad Las pruebas se centrarán en la integración de la API, de tal forma que se comprobará el estado de la lista mostrada al usuario tras realizar las distintas operaciones disponibles. &&
    \labelitemi \quad Se documentará apropiadamente el código del API expuesto al usuario.
    \end{tabulary}
\end{table}
 
 
\begin{table}[hbtp]
    \footnotesize
    \centering
    \settowidth\tymin{\textbf{Funcionalidad}}
    \setlength\extrarowheight{5pt}
    \begin{tabulary}{\textwidth}{ C L }
        \textbf{F3} & incluye un sistema de redirección para que el usuario pueda monetizar.
        \\
    \hline
    
    Cumplimiento &
    \begin{itemize}
        
        \item Al acceder a una de las URIs generadas, será redirigido a una página de espera, donde tendrá que esperar 5 segundos para continuar a la URI sin "shortening", solo con las URL asociadas a usuarios de la plataforma.
        
    \end{itemize} &
        
    Escalabilidad\\ &
    5:
    \begin{itemize}
        \item No necesita escalar.
        
    \end{itemize} 
    10:
    \begin{itemize}
        \item Cumple el nivel 5
		
        \item La petición de la URL es asincrónica y está no se devolverá hasta que halla pasado el tiempo esperado.
        \item Se evitará la obtención de la URL antes del tiempo de espera.
        
    \end{itemize} 
    15:
    \begin{itemize}
        \item Cumple el nivel 10
        
        \item La petición de "shortening" es gestionada primero por un hilo, el cual añade a una cola, global en el servidor Web, la petición en cuestión.
        
        \item Otro hilo distribuye las peticiones de la cola entre una serie de hilos workers, los cuales realizan las funciones de comprobación de validez, generación de URIs y guardado en el servidor Web.
        
        \item La página de espera al acceder a una URI acortada contiene anuncios de verdad, los cuales generan dinero.
	\end{itemize} &
 
    Profesionalidad & 
    \begin{itemize}
        \item La API de las peticiones HTML de creación de nuevas URIs debe estar documentada.
	    \item Debe haber una serie de pruebas automáticas sobre la API de creación de nuevas URIs.
    \end{itemize}
 
    \end{tabulary}
\end{table}
 
\begin{table}[hbtp]
    \footnotesize
    \centering
    \settowidth\tymin{\textbf{Funcionalidad}}
    \setlength\extrarowheight{5pt}
    \begin{tabulary}{\textwidth}{ C L }
        \textbf{F4} & La aplicación verificar a que la uri que el usuario quiere crear es alcanzable
        \\
    \hline
    
    Cumplimiento &
    \begin{itemize}
        \item La aplicación verificará al crear la URL que esta es alcanzable (se
        puede hacer una petición http GET que en un tiempo razonable
        devuelve una respuesta 200).
 
        \item Si esta funcionalidad se implementa con un API Web su
        implementación será consistente con la semántica de http
 
        \item POST /link comprueba que la URL es alcanzable y si no lo es devuelve
        una respuesta con un error 400.
 
        \item GET /{hash} nunca podrá redirigir a una URL que antes no se haya
        comprobado que sea alcanzable. Si el hash está registrado pero la
        URL de destino no está verificada se devuelve una respuesta con un
        error 400.
 
        \item Si alguna de las dos peticiones anteriores devuelve errores de tipo
        400 por más de un motivo deberá devolver un objeto JSON
        especificando el motivo concreto del error del usuario.
    \end{itemize} &
        
    Escalabilidad\\ &
    5:
    \begin{itemize}
        \item No necesita escalar
        
    \end{itemize} 
    10:
    \begin{itemize}
        \item Cumple el nivel 5
		
		\item El servidor devuelve inmediatamente la URL acortada de forma
        condicional ya que su verificación la va a realizar algún proceso
        asíncrono
		
        \item La URL acortada no estará operativa mientras que la URL no sea
        validada
    \end{itemize} 
    15:
    \begin{itemize}
        \item Cumple el nivel 10
        
        \item El trabajo asíncrono de verificación estará gestionado por una
        cola de tareas en la que se publicará trabajos
        
        \item Una máquina cliente se subscribirá a la cola de tareas para
        realizar trabajos de verificación
        
        \item El resultado de la verificación se devolverá por otra cola de
        tareas
 
 
	\end{itemize} &
 
    Profesionalidad & 
    \begin{itemize}
        \item La API de las peticiones HTML de creación de nuevas URIs debe estar documentada.
	    \item Debe haber una serie de pruebas automáticas sobre la API de creación de nuevas URIs.
    \end{itemize}
 
    \end{tabulary}
\end{table}
 
\end{document}
 
